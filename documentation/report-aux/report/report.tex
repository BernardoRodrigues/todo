\documentclass[a4paper,openright,twoside,11pt]{report}
\usepackage[utf8]{inputenc}
\usepackage[english]{babel}

\begin{filecontents}{rp.bib}
	@inproceedings{scale,
		author = {Bondi, André},
		year = {2000},
		month = {01},
		pages = {195-203},
		title = {Characteristics of Scalability and Their Impact on Performance},
		journal = {Proceedings Second International Workshop on Software and Performance WOSP 2000},
		doi = {10.1145/350391.350432}
	},
	@techreport{jwt,
		title = {JSON Web Token (JWT)},
		author = {Michael B. Jones, John Bradley, Nat Sakimura},
		howpublished = {Internet Requests for Comments},
		url = {https://tools.ietf.org/html/rfc7519}
		year = {2015},
		number = 7519,
		type = {RFC},
		publisher = {RFC Editor},
		institution = {RFC Editor},
		issn = {2070-1721},
		month = {5}
	},
	@article{
		repositorypattern,
		title = {An Introduction to Software Architecture},
		author = {David Garlan, Mary Shaw},
		journal = {School of Computer Science Carnegie Mellon University},
		year = 1994,
		pages = "12--13",
	},
	@techreport{pushprotocol,
		title = {Generic Event Delivery Using HTTP Push},
		author = {Martin Thomson, Elio Damaggio, Brian Raymor},
		howpublished = {Internet Requests for Comments},
		url = {https://tools.ietf.org/html/rfc8030}
		year = {2016},
		number = 8030,
		type = {RFC},
		publisher = {RFC Editor},
		institution = {RFC Editor},
		issn = {2070-1721},
		month = {12}
	}
\end{filecontents}
\usepackage[acronyms, nonumberlist]{glossaries}
\newacronym[plural={Json Web Tokens}]{jwt}{JWT}{Json Web Token}
\newacronym[plural={Application Programming Interfaces}]{api}{API}{Application Programming Interface}
%{long}{jwt}{name={JWT},description={JSON Web Token}}
\usepackage{lipsum} % gerador de texto
\usepackage{graphicx}
\usepackage{url}
%\usepackage[Algoritmo]{algorithm}
\usepackage{ragged2e}
\usepackage[export]{adjustbox}
\usepackage{indentfirst}
\usepackage{xr}
\usepackage{hyperref}
\bibliographystyle{plain}

% Definições das dimensões das páginas
\setlength{\textheight}{24.00cm}
\setlength{\textwidth}{15.50cm}
\setlength{\topmargin}{0.35cm}
\setlength{\headheight}{0cm}
\setlength{\headsep}{0cm}
\setlength{\oddsidemargin}{0.25cm}
\setlength{\evensidemargin}{0.25cm}

\makeglossaries


% Página inicial (capa)
\title{
	\vspace{-50mm}
	\begin{minipage}[l]{\textwidth}
		\hspace{-20mm}\resizebox{75mm}{!}{\includegraphics{./images/logoISEL.png}}\\
	\end{minipage}\\[20mm]
	{\bf todo}
}

% Nome dos autores (um por linha)
\author{
	\begin{tabular}{ll}
		& Bernardo Rodrigues
\end{tabular}}

\date{
	\begin{tabular}{ll}
		{Supervisor:} & Nuno Leite \\
	\end{tabular}\\[10mm]
	% Deixar o indicador respetivo em função da versão do relatório.
	Project report on Projecto e Seminário\\
	Graduation on Software Engineering and Computer Science\\[20mm]
	July 2021}

\begin{document}
	\renewcommand{\abstractname}{\vspace TODO{-\baselineskip}}
	\thispagestyle{empty}
	\maketitle
	
	\baselineskip 18pt % line spacing: 12pt for single, 18pt for 1 1/2, and 24pt for double spacing
	
	\newpage
	\thispagestyle{empty}
	% Fim da contracapa
	
	% Página com identificação completa (número e nome) e assinaturas do(s) estudante(s) e do(s) orientador(es)
	
	\setcounter{page}{1}
	
	
	% Página de resumo em Português
	\cleardoublepage
	
	\printglossary[type=\acronymtype]
	
	%web push protocol https://tools.ietf.org/html/draft-ietf-webpush-protocol-12
	
	\include{abstract}
	
	\chapter{Introduction}

In today’s fast and complex world people try to schedule days or weeks. People create small
reminders, either electronically or physically, to do something, whatever that may be. Those
little notes we give to ourselves, those ”to-do’s” are often forgotten and not fully fulfilled.
There exists software applications in the market that partially accomplishes this goal. They will often remind the user of the task that needs to be done only using the time frame the user gave. This creates a problem where the individual makes a reminder,
does not do it and the application will not keep notifying the user.\\
As such the proposed solution will attempt to mend that by creating an environment where a reminder is easily
created but hardly forgotten or skipped over. The solution consists of a web application divided into three different components, a front-end web app, a back-end web \gls{api} and a database. This application will allow the users to create
an account and then make reminders. The users will be notified by the application until they mark said reminder as done.\\
The notification is a necessity to ensure the user fulfills the reminder. Since the user will be notified until the reminder is marked as "done" the person will be reminded every day until the completion of the task.

For this purpose a set of mandatory requirements was outlined:
\begin{itemize}
	\item Account creation, login, logout and account deletion;
	\item Reminder creation, update and deletion;
	\item Notifications;
\end{itemize}

Some optional requirements were also outline to enrich the application:
\begin{itemize}
	\item Translations
	\item Priority notifications
\end{itemize}

This report will outline the process of the application development, the choices made and the work
that remains to be done.
	
	\chapter{State of the art}
	
	
	The need to organize our daily lives isn't a new concept and as such, to-do applications are not new. A few examples of some of these are \textit{Todoist \cite{todoist}}, and Google Calendar's reminder system \cite{google-calendar}.
	
	The main difference between these applications and the one being developed in this project is the image recognition component as well as the prioritization algorithm, although the aforementioned solutions do have mobile applications which make notifications more effective.
	
	\chapter{Architecture}

	%https://martinfowler.com/articles/microservices.html
	%
	
	The architecture for the API was thoroughly researched and studied. At first this project was gonna use a monolithic architecture \cite{monolith} where the whole API would be implement on a single server as seen in the image below.
	%TODO add image of monolith arch
	
	This was later changed due to the fact that it wasn't as easily scalable \cite{scale}. After this it was decided that the API would be implemented using a \textit{microservice}\cite{microservice} architecture. This allows for a more select scalability. When needed, more instances of the same services can be launched and improve overall performance.
	
	There will be different services:
	\begin{enumerate}
		\item User
		\item To-do
		\item Task
	\end{enumerate}

	%TODO add image here add image of micro-service arch
	
	There was also the idea of combining both the monolithic and micro-service architecture in a mixture of both patterns. This idea was later scrapped and the micro-service system was then chosen.
	
	%TODO add image of monolith - microservice arch
	
	Each of these services are independent from one another and self sufficient, each with it's own responsibilities. 
	
	%https://tools.ietf.org/html/rfc7519
	%https://www.rfc-editor.org/rfc/pdfrfc/rfc7519.txt.pdf
	
	% https://www.cs.cmu.edu/afs/cs/project/vit/ftp/pdf/intro_softarch.pdf
	
	\section{User}
	The user micro-service will have the operations needed for an user to utilize the application. These operations consist of the sign up, login, logout and, delete.
	The micro-service will use \gls{jwt} \cite{jwt} as a way to make sure the users only access their information.
	The service is split into two different areas, the repository\cite{repositorypattern} that accesses the database and ,the routing.
	
	\section{To-do}
	The to-do micro-service handles all the operations necessary to manage the to-do's. These operations embedded in this service are to-do creation, deletion and update.
	This micro-service will interact with the user service to validate and extract the \gls{jwt} information.
	
	\section{Task}
	The task micro-service has three main objectives. It offers a way to subscribe and unsubscribe users to receive notifications as well as checking the database periodically to ensure the notifications are sent to the subscribed users. The notifications will be sent according to the push protocol\cite{pushprotocol}.
	 
	
	\chapter{Database and Data Modeling}

	The database model is described, in this chapter. It's a fairly simple model although additions can be made to augment and improve the application's features.

	\begin{figure}[h!]
		\includegraphics[width=17cm,scale=0.6]{./images/data_model_2do}
		\caption{Data Model}
	\end{figure}

	A user can have multiple to-do's and subscriptions (tasks). The need for multiple subscriptions stems from the fact that the user can have multiple instances of the application running in different desktops.\\
	There also is a priority system which hasn't yet been implemented. In the future each to-do will have a priority with a specific value and this value will be used to calculate how many times the user will be notified.
	
	
	\chapter{Web \gls{api}}
	
	%add api doc
	%communication flow in the api
	%protocols
	%security
	%packages used
	%etc
	
	The \gls{rest} web \gls{api} is composed of three micro-services, an \gls{api} gateway and a neural network\cite{neuralnetwork}.\\
	The micro-services all have different responsibilities but they mostly follow the same pattern. Each service is divided into two main layers, the routing and the \gls{dal} which is comprised of a repository.
	The whole \gls{api} was built utilizing NodeJS\cite{node} and Typescript\cite{typescript}.
	All services use \textit{node-postgres}\cite{node-postgres} to interact with the database.
	The gateway and some services utilize the \textit{axios}\cite{axios} package to send HTTP requests.
	
	
	\section{\gls{api} Gateway}
	
		The \gls{api} gateway's purpose is to receive incoming requests and redirect them to the intended services. To make it follow the micro-service pattern it requires it to be stateless.
		The gateway exposes a \gls{rest} \gls{api} to be used by the front-end applications. 
		
		
	\section{User Service}
		
		The user service is responsible for all the operations regarding user accounts as well as verifying each request. The service exposes a route for other services to verify the \gls{jwt} authenticity. This route is only known by the other services and not available to the gateway.
	
	\section{To-do Service}
		The to-do service serves the purpose of creating, updating and canceling to-do's. The service exposes a route for the task service so it can get all undone reminders. It also utilizes the verification route exposed by the user service as previously mentioned.
	
	\section{Task Service}
		The task service periodically checks for undone to-do's. It does by sending an HTTP request to the to-do service to get all undone reminders. It also utilizes the user service to verify the authenticity of the requests. 
		When the server is starting it schedules an event using the Node Schedule package\cite{nodeschedule} to periodically make these requests. As of right now it only checks every day at a specific time but eventually it will be able to receive a granularity value and be dynamic.
		This service also is the one that sends the notifications to the users. It does this by using the Web Push package\cite{webpush}. 
		If there is a failure when sending the notification it will erase the endpoint stored in the database as it probably means the user left the session or the endpoint is no longer available.
		The task service utilizes VAPID\cite{vapid} keys. These keys are essential to utilize the Web Push Protocol.
	
	\bibliography{rp.bib}
	
\end{document}