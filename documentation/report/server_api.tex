\chapter{Web \gls{api}}
	
	%add api doc
	%communication flow in the api
	%protocols
	%security
	%packages used
	%etc
	
	The \gls{rest} web \gls{api} is composed of three micro-services, an \gls{api} gateway and a neural network\cite{neuralnetwork}.\\
	The micro-services all have different responsibilities but they mostly follow the same pattern. Each service is divided into two main layers, the routing and the \gls{dal} which is comprised of a repository.
	The whole \gls{api} was built utilizing NodeJS\cite{node} and Typescript\cite{typescript}.
	All services use \textit{node-postgres}\cite{node-postgres} to interact with the database.
	The gateway and some services utilize the \textit{axios}\cite{axios} package to send HTTP requests.
	
	
	\section{\gls{api} Gateway}
	
		The \gls{api} gateway's purpose is to receive incoming requests and redirect them to the intended services. To make it follow the micro-service pattern it requires it to be stateless.
		The gateway exposes a \gls{rest} \gls{api} to be used by the front-end applications. 
		
		
	\section{User Service}
		
		The user service is responsible for all the operations regarding user accounts as well as verifying each request. The service exposes a route for other services to verify the \gls{jwt} authenticity. This route is only known by the other services and not available to the gateway.
	
	\section{To-do Service}
		The to-do service serves the purpose of creating, updating and canceling to-do's. The service exposes a route for the task service so it can get all undone reminders. It also utilizes the verification route exposed by the user service as previously mentioned.
	
	\section{Task Service}
		The task service periodically checks for undone to-do's. It does by sending an HTTP request to the to-do service to get all undone reminders. It also utilizes the user service to verify the authenticity of the requests. 
		When the server is starting it schedules an event using the Node Schedule package\cite{nodeschedule} to periodically make these requests. As of right now it only checks every day at a specific time but eventually it will be able to receive a granularity value and be dynamic.
		This service also is the one that sends the notifications to the users. It does this by using the Web Push package\cite{webpush}. 
		If there is a failure when sending the notification it will erase the endpoint stored in the database as it probably means the user left the session or the endpoint is no longer available.
		The task service utilizes VAPID\cite{vapid} keys. These keys are essential to utilize the Web Push Protocol.